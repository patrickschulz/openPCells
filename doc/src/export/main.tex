\usetheme{opc}
\usepackage[]{appendixnumberbeamer}

% Font
\RequirePackage{mathpazo}
%\RequirePackage{fontspec}
%\setsansfont{Beteckna}
%\setsansfont{Noto Sans}
%\setsansfont{Futura LT}

%% Graphics
%\RequirePackage{shellesc}
\RequirePackage{tikz}
\usetikzlibrary{
%    arrows, 
%    arrows.meta,
%    backgrounds,
%    calc,
%    chains,
%    decorations,
%    decorations.markings,
%    external,
%    fit,
%    matrix,
    patterns,
%    positioning,
%    scopes,
%    shapes,
%    shapes.geometric,
%    fadings
}
%\tikzexternalize[prefix=tikzfigures/]
\input{tikzsettings}
%% Pgfplots
%\RequirePackage{pgfplots}
%\usepgfplotslibrary{patchplots, groupplots}
%\input{pgfplotssettings}
%% Circuits symbols
%\input{circuit_symbols}

% Code listings
\usepackage{listings}
\lstset{
    breaklines = true,
    %breakatwhitespace = true,
    language=[5.3]Lua,
    backgroundcolor = \color{blue!10!white},
    basicstyle = \small\ttfamily,
    keywordstyle = \color{blue},
    commentstyle = \color{red},
    stringstyle = \color{green!70!black},
    showstringspaces = false,
    tabsize = 4,
    gobble = 8,
    numbers = none,
    emph = {
        layout,
        parameters,
        config,
        pointarray,
        path,
        geometry,
        geometry.rectangle, 
        geometry.rectanglebltr, 
        geometry.polygon, 
        geometry.via, 
        geometry.viabltr, 
        geometry.contact, 
        geometry.contactbltr, 
        geometry.path, 
        pcell.setup, 
        pcell.add_cell_reference, 
        pcell.create_layout, 
        pcell.process_args, 
        pcell.check_args,
        pcell.add_parameter,
        pcell.add_parameters,
        pcell.get_parameters,
        object.create,
        object.translate,
        object.add_child,
        object.merge_into,
        get_anchor,
        move_anchor,
        translate,
        merge_into,
        flipx,
        flipy,
        add_child,
        add_child_reference,
        add_child_link,
        generics,
        generics.metal,
        generics.via,
        generics.contact,
        generics.other,
        generics.mapped,
        util.xmirror,
        util.make_insert_xy,
        posvals,
        even,
        odd,
        interval,
        set,
        inf
    },
    emphstyle = \color{blue!60!green}\bfseries,
    %belowskip=-0.2\baselineskip
}
\lstnewenvironment{luacode}{}{}
\newcommand{\luainline}[1]{\lstinline!#1!}
%\lstnewenvironment{shellcode}{\lstset{keywordstyle = \relax, commentstyle = \relax, stringstyle = \relax, identifierstyle = \relax, breakautoindent = false}}{}

% Units
\RequirePackage[per-mode=fraction, range-phrase=\ ---\ , exponent-product = \cdot]{siunitx}
\DeclareSIUnit{\dBc}{dBc}
\DeclareSIUnit{\dBchz}{dBc/Hz}

% Maths
\usepackage{amsmath}
\usepackage{IEEEtrantools}

% Tables
\usepackage{tabu, array, longtable, booktabs, makecell}

\usepackage{csquotes}


\title{OpenPCells}
\subtitle{Writing Custom Exports}
\author{Patrick Kurth}

\begin{document}
\maketitle
\begin{abstract}
    \noindent This is the official documentation of the OpenPCells project. It is split in several different files for clarity. 
    This document provides an overview of the creation of custom export types.
    If you are looking for a general overview of the project and how to use it, start with the user guide, which also contains a tutorial for getting started quickly. 
    If you are looking for a guide and documentation on the creation of parametrized cells, consult the celldesign manual.
    If you want to now more about the technical details and implementation notes, look into the technical documentation.
\end{abstract}

\section{Overview}
Exports in openPCells work by defining functions that write specific shapes/objects such as rectangles or polygons.
The functions that need to be defined follow closely the way layouts are represented in opc.
Some functions (such as writing rectangles) are elementary and are mandatory, other are optional (such as functions dealing with cell hierarchies).
The calling environment of an export makes sure to reduce the layout to a representation that the export can understand (for example flattening layout hierarchies).
In total, 16~different functions can be defined, but only 4 are mandatory.

In the following, all export functions (mandatory and optional) will be discussed in detail and some basic best practices regarding the writing of export types will be given.
All viewings of export types will be focused on lua exports.
C exports follow a similar fashion, but have more freedom in their processing of output data.
The specific differences are shown in section~\ref{sec:cspecial}.

\section{Export Functions}
\subsection{initialize}
%-- * optional *
%-- make arbitrary calculations that are needed for this export
%-- the function gets the minimum and maximum coordinates (x and y)
%function M.initialize(minx, maxx, miny, maxy)
%end
%
%-- * mandatory *
%-- function which defines how the export gets its layer
%-- usually pretty simple, as most of the work is already done in the technology translation
%-- example from the GDS export: return { S:get_lpp():get().layer, purpose = S:get_lpp():get().purpose }
\subsection{finalize}
%function M.finalize()
%    -- example for string return
%    -- most exports work like this, but some need to do something else
%    return table.concat(__content)
%end
%
%-- * mandatory *
%-- provides the file ending of the generated layout (e.g. returns "gds" for the gds export)
\subsection{get\_extension}
%function M.get_extension()
%    return "TEMPLATE"
%end
%
%-- * optional *
%-- provides a different name to be used for the technology translation
%-- usually, each export needs their one layer definitions in the technology layermap,
%-- but some exports can reuse other data, such as the OASIS export, which can reuse 
%-- layer names from GDS
\subsection{get\_techexport}
%function M.get_techexport()
%    return "OTHERTEMPLATE"
%end
%
%-- * optional *
%-- callback for export command line options
%-- before the export starts, this function (if present) is called with the collected
%-- command line arguments.
\subsection{set\_options}
%function M.set_options(opt)
%end
%
%-- * optional *
%-- function called at the start of the export, data will be written before other data
\subsection{at\_begin}
%function M.at_begin()
%end
%
%-- * optional *
%-- function called at the end of the export, data will be written after all other data
\subsection{at\_end}
%function M.at_end()
%end
%
%-- * optional *
%-- function called at the start of a cell
%-- usefull for exports that support hierarchies, but some exports such as GDS need this 
%-- also in the case of flat layouts
\subsection{at\_begin\_cell}
%function M.at_begin_cell(cellname)
%end
%
%-- * optional *
%-- counterpart to at_begin_cell. Called AFTER the cell is written
\subsection{at\_end\_cell}
%function M.at_end_cell()
%end
%
%-- * mandatory *
%-- how to write a rectangle
\subsection{write\_rectangle}
%function M.write_rectangle(layer, bl, tr)
%end
%
%-- * optional *
%-- how to write a triangle
\subsection{write\_triangle}
%function M.write_triangle(layer, pt1, pt2, pt3)
%end
%
%-- * sort-of mandatory *
%-- how to write a polygon
%-- if this is not present but write_triangle is provided,
%-- polygons are triangulated and written by write_triangle
\subsection{write\_polygon}
%function M.write_polygon(layer, pts)
%end
%
%-- * optional *
%-- how to write a path
%-- if not present, the shape will be converted accordingly
%-- (to a single rectangle if possible, otherwise to a polygon)
\subsection{write\_path}
%function M.write_path(layer, pts, width)
%end
%
%-- * optional *
%-- how to write a cell reference (a child in opc terminology). Needed for hierarchies
\subsection{write\_cell\_reference}
%function M.write_cell_reference(identifier, x, y, orientation)
%end
%
%-- * optional *
%-- how to write an array of cell reference
\subsection{write\_cell\_array}
%function M.write_cell_array(identifier, x, y, orientation, xrep, yrep, xpitch, ypitch)
%end
%
%-- * optional *
%-- how to write a named for layout topology data (e.g. LVS)
\subsection{write\_cell\_port}
%function M.write_port(name, layer, where)
%end

\section{C Export Specialities}\label{sec:cspecial}

\end{document}

% vim: ft=tex
