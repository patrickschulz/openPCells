\documentclass[parskip=half]{scrartcl}

\usepackage{tikz}
\usetikzlibrary{positioning, shapes, patterns}
\tikzset{
    state/.style = {draw, align=center, minimum width = 2.1cm, rectangle split, rectangle split parts = 3, thick, minimum height = 5cm},
    tip/.style = {thick, ->, >=stealth}
}
\usepackage{wrapfig}

\usepackage{fontspec}
\setmainfont{Linux Libertine O}

\usepackage{listings}
\lstset{
    language=[5.3]Lua,
    backgroundcolor = \color{blue!10!white},
    basicstyle = \small\ttfamily,
    keywordstyle = \color{blue},
    commentstyle = \color{red},
    stringstyle = \color{green!70!black},
    showstringspaces = false,
    tabsize = 4,
    gobble = 4,
    emph = {
        pointarray,
        path,
        layout.rectangle, 
        layout.multiple, 
        layout.path, 
        pcell.setup, 
        pcell.process_args, 
        pcell.check_args,
        generics,
        generics.metal,
        generics.via,
        generics.contact,
        generics.other,
        generics.mapped,
    },
    emphstyle = \color{blue!60!green}\bfseries,
    belowskip=-0.2\baselineskip
}
% code listings
%\lstnewenvironment{lua}{\lstset{language=lua}}{}
\lstnewenvironment{shell}{\lstset{language=bash, keywordstyle = \relax, commentstyle = \relax}}{}
\newcommand{\shellinline}[1]{\lstinline!#1!}
% API documentation commands
\newcommand{\param}[1]{\textcolor{blue!60!green}{\bfseries\ttfamily #1}}
\newenvironment{apifunc}[1]{\par\hspace*{-2em}\lstinline!#1!\leftskip=2em\par}{\par}

\usepackage{minted}
\newminted{lua}{fontsize = \small}
\newmintedfile[lualisting]{lua}{autogobble, fontsize = \small}
%\usemintedstyle{manni} % A colorful style, inspired by the terminal highlighting style.
%\usemintedstyle{igor} % Pygments version of the official colors for Igor Pro procedures.
\usemintedstyle{lovelace} % The style used in Lovelace interactive learning environment. Tries to avoid the "angry fruit salad" effect with desaturated and dim colours.
%\usemintedstyle{xcode} % Style similar to the Xcode default colouring theme.
%\usemintedstyle{vim} % Styles somewhat like vim 7.0
%\usemintedstyle{autumn} % A colorful style, inspired by the terminal highlighting style.
%\usemintedstyle{abap} % 
%\usemintedstyle{vs} % 
%\usemintedstyle{rrt} % Minimalistic "rrt" theme, based on Zap and Emacs defaults.
%\usemintedstyle{native} % Pygments version of the "native" vim theme.
%\usemintedstyle{perldoc} % Style similar to the style used in the perldoc code blocks.
%\usemintedstyle{borland} % Style similar to the style used in the borland IDEs.
%\usemintedstyle{arduino} % The Arduino® language style. This style is designed to highlight the Arduino source code, so exepect the best results with it.
%\usemintedstyle{tango} % The Crunchy default Style inspired from the color palette from the Tango Icon Theme Guidelines.
%\usemintedstyle{emacs} % The default style (inspired by Emacs 22).
%\usemintedstyle{friendly} % A modern style based on the VIM pyte theme.
%\usemintedstyle{monokai} % This style mimics the Monokai color scheme.
%\usemintedstyle{paraiso-dark} % 
%\usemintedstyle{colorful} % A colorful style, inspired by CodeRay.
%\usemintedstyle{murphy} % Murphy's style from CodeRay.
%\usemintedstyle{bw} % 
%\usemintedstyle{pastie} % Style similar to the pastie default style.
%\usemintedstyle{rainbow_dash} % A bright and colorful syntax highlighting theme.
%\usemintedstyle{algol_nu} % 
%\usemintedstyle{paraiso-light} % 
%\usemintedstyle{trac} % Port of the default trac highlighter design.
%\usemintedstyle{default} % The default style (inspired by Emacs 22).
%\usemintedstyle{algol:} % 
%\usemintedstyle{fruity} % Pygments version of the "native" vim theme.

\usepackage{siunitx}
\usepackage{csquotes}

\frenchspacing

